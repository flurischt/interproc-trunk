\documentclass[a4paper,11pt]{article}
\usepackage{hyperlatex}
\usepackage{amssymb,stmaryrd,xspace}
%\usepackage{frames}
\htmlcss{style.css}
\htmltitle{Interproc analyzer for recursive programs with numerical variables}
\htmlpanel{1}
\setcounter{htmlautomenu}{2}
\setcounter{htmldepth}{3}

\W\usepackage{sequential}
\W\usepackage{mypanels}
\htmlpanelfield{Contents}{hlxcontents}


\T\boldmath
\T\usepackage{fullpage}
%\T\setlength{\parskip}{0ex}
%\T\setlength{\parindent}{1em}

\newcommand{\sem}[1]{\llbracket #1 \rrbracket}

\usepackage{color}
\definecolor{prog}{rgb}{0,0.4,0.1}
\definecolor{progkeyword}{rgb}{0.8,0,0}
\definecolor{token}{rgb}{0,0,1}
\definecolor{emph}{rgb}{0,0.1,0.6}
\definecolor{example}{rgb}{.8,0,1}
\newcommand{\pkw}[1]{\textcolor{progkeyword}{#1}}
\newcommand{\pco}[1]{\textcolor{black}{-- #1}}
\W\renewcommand{\emph}[1]{\textcolor{emph}{#1}}
\newcommand{\sx}[1]{\textcolor{prog}{\texttt{<}#1\texttt{>}}}
\newcommand{\kw}[1]{\textcolor{progkeyword}{\textbf{#1}}}
\newcommand{\tk}[1]{\textcolor{token}{\texttt{#1}}}
\newcommand{\sor}{\texttt{~|~}}

\newenvironment{syntax}
{
\T\medskip
\begin{tabular}{rcll}
}
{
\end{tabular}
\T\par\medskip
}
\newenvironment{prog}
{\begin{example}\color{prog}}
{\end{example}\par}

\newcommand{\ocaml}{\xlink{OCaml}{http://www.caml.org}\xspace}
\newcommand{\nbac}{\xlink{NBac}{../nbac/index.html}\xspace}
\newcommand{\gmp}{\xlink{GMP}{http://www.swox.com/gmp/}\xspace}
\newcommand{\mpfr}{\xlink{MPFR}{http://www.mpfr.org/}\xspace}
\newcommand{\camlidl}{\xlink{CamlIDL}{http://caml.inria.fr/camlidl/}\xspace}
\newcommand{\subversion}{\xlink{subversion}{http://subversion.tigris.org/}\xspace}
\newcommand{\ppl}{\xlink{Parma Polyhedra Library}{http://www.cs.unipr.it/ppl/}\xspace}
\newcommand{\apron}{\xlink{APRON}{http://apron.cri.ensmp.fr/library/}\xspace}

\title{Interproc analyzer for recursive programs with numerical variables}
\date{}
\author{Bertrand Jeannet}

\begin{document}
\maketitle
\tableofcontents

% **********************************************************************
\section{Invoking Interproc }
% **********************************************************************

The executable (\texttt{interproc} or \texttt{interproc.opt} is
invoked as follows:
\begin{example}
  interproc <options> <inputfile>
\end{example}
The input file should be a valid \link{``Simple''}{sec:simple} program.
The options are:
\begin{description}
\item [\texttt{-debug <int>}] debug level, from 0 (lowest) to 4
  (highest). Default is 0.
\item [\texttt{-domain
    \{box|octagon|polka|polkastrict|polkaeq|ppl|pplstrict|pplgrid|polkagrid\}}]
  abstract domain to use (default: polka). All domains supported
  by the \xlink{Apron}{http://apron.cri.ensmp.fr/library/}
  library can be specified:
  \begin{quote}
    \begin{tabular}{l|l}
      box & intervals \\
      octagon & octagons \\
      polka,ppl & topologically closed convex polyhedra \\
      polkastrict,pplstrict & possibly non-closed convex polyhedra \\
      polkaeq & linear equalities \\
      pplgrid & linear congruences \\
      polkagrid & reduced product of linear congruences and Polka convex polyhedra
    \end{tabular}
  \end{quote}
\item [\texttt{-depth <int>}] depth of recursive iterations
  (default 2, may only be more). See
  \xlink{Fixpoint}{http://pop-art.inrialpes.fr/people/bjeannet/bjeannet-forge/fixpoint/}
  library documentation.
\item [\texttt{-guided <bool>}] if true, guided analysis of Gopand
  and Reps (default: false). See
  \xlink{Analyzer}{http://pop-art.inrialpes.fr/people/bjeannet/analyzer/}
\item [\texttt{-widening <bool><int><int><int>}] specifies usage
  of widening first heuristics, delay and frequency of widening,
  and nb. of descending steps (default: false 1 1 2). See
  \xlink{Fixpoint}{http://pop-art.inrialpes.fr/people/bjeannet/bjeannet-forge/fixpoint/}
\item [\texttt{-analysis <('f'|'b')+>}] sequence of forward and backward analyses to perform (default "f")
\end{description}

% **********************************************************************
\section{The ``Simple'' language}\label{sec:simple}
% **********************************************************************

Prorgram Sample:
{
\begin{prog}\footnotesize
/* Procedure definition */
proc MC(n:int) returns (r:int)
var t1:int, t2:int;
begin
  if (n>100) then
     r = n-10;
  else 
     t1 = n+11;
     t2 = MC(t1);
     r = MC(t2);
  endif;
end
/* Main procedure */
var a:int, b:int;
begin
  b = MC(a);
end
\end{prog}
}
                                
% ======================================================================
\xname{manual_syntax}
\subsection{Syntax and informal semantics}
% ======================================================================

% ----------------------------------------------------------------------
\subsubsection{Program.}\T~
% ----------------------------------------------------------------------

\begin{syntax}
  \sx{program} & ::= & \sx{procedure\_def}$^*$ & definition of procedures\\
  && [\pkw{var} \sx{vars\_decl} \tk{;}] & local variables of the main procedure\\
  && \pkw{begin} \sx{statement}$^+$ \pkw{end} ~~~ & body of the main procedure
\end{syntax}

\begin{syntax}
\sx{procedure\_def} & ::= & \pkw{proc} id \tk{(} \sx{vars\_decl} \tk{)} \pkw{returns} \tk{(} \sx{vars\_decl} \tk{)} & signature, with names of \\
  &&& input and output formal parameters \\
  && [\pkw{var} \sx{vars\_decl} \tk{;}] & (other) local variables \\
  &&\pkw{begin} \sx{statement}$^+$ \pkw{end} & body
\end{syntax}
A procedure definition first declares a list of formal input
parameters, then a list of formal output parameters (which are
local variables that defines the tuple of returned values), then a
list of local variables, and then a body. A procedure ends with
its last statement.

\begin{syntax}
  \sx{vars\_decl} & ::= & $\epsilon$ \sor id \tk{:} \sx{type} (\tk{,} id \tk{:} \sx{type})$^*$ \\
\sx{type} & ::= & \pkw{int} \sor \pkw{real} \\
\end{syntax}
Variables are either of integer type, or of real type. Real
variables can behave as floating-point variables by modifying the
exact semantics of arithmetic operations (see below).

% ----------------------------------------------------------------------
\subsubsection{Instructions.}\T~
% ----------------------------------------------------------------------

\begin{syntax}
  \sx{statement} &::=& \pkw{skip} \tk{;} \\
  &\sor& \pkw{halt} \tk{;} \\
  &\sor& \pkw{fail} \tk{;} \\
  &\sor& \pkw{assume} \sx{bexpr} \tk{;} \\
  &\sor& id \tk{=} \pkw{random} \tk{;} \\
  &\sor& id \tk{=} \sx{nexpr} \tk{;} \\
  &\sor& id \tk{=} id \tk{(} \sx{var\_list} \tk{)} \tk{;} \\
  &\sor& \tk{(} \sx{var\_list} \tk{)} \tk{=} id \tk{(} \sx{var\_list} \tk{)} \tk{;} \\
  &\sor& \pkw{if} \sx{bexpr} \pkw{then} \sx{statement}$^+$ [\pkw{else} \sx{statement}$^+$ ] \pkw{endif} \tk{;} \\
  &\sor& \pkw{while} \sx{bexpr} \pkw{do} \sx{statement}$^+$ \pkw{done} \tk{;} \\
  \sx{var\_list} &::=& $\epsilon$ \sor id (\tk{,} id)$^*$
\end{syntax}

The \texttt{skip} operation does nothing.  The \texttt{halt}
operation halts the program execution. The \texttt{fail} has
exactly the same operational semantics, but, in addition,
indicates a final control point for backward analysis (see below).

The \texttt{assume expr} instruction is equivalent to \texttt{if
  expr then skip; else halt;} . It may be used to abstract
non-deterministic assignements. For instance, the
non-deterministic assignement of a value between 0 and 2 to a
variable \texttt{x} can be written as 
\begin{prog}
  x = random; 
  assume x>=0 and x<=2;
\end{prog}

A call to a procedure with a single return value can be written
either as \texttt{x = P(...)} or \texttt{(x)=P(...)}. A call to a
procedure with multiple return values is written as \texttt{(x,y)
  = P(a,b)}. Input parameters are passed by value. Formal input
parameters in a procedure are assumed immutable.

Conditional and iteration constructions have the usual semantics.

% ----------------------------------------------------------------------
\subsubsection{Expressions.}\T~
% ----------------------------------------------------------------------

\paragraph{Boolean expressions.}\T~

\begin{syntax}
  \sx{bexpr} &::=& \pkw{true} \sor \pkw{false} \sor \pkw{brandom} \\
  &\sor& \sx{constraint} \\
  &\sor& \pkw{not} \sx{bexpr} \sor \sx{bexpr} \pkw{or} \sx{bexpr} \sor \sx{bexpr} \pkw{and} \sx{bexpr} \\
  &\sor& \tk{(} \sx{bexpr} \tk{)} \\
  \sx{constraint} &::=& \sx{nexpr} (\tk{=} \texttt{|} \tk{>=} \texttt{|} \tk{>} \texttt{|} \tk{<=} \texttt{|} \tk{<}) \sx{nexpr}
\end{syntax}

Atoms of Boolean expressions are Boolean constants, the unknown
value denoted by \pkw{brandom}, and constraints over numerical
variables.

\paragraph{Numerical expressions.}\T~

\begin{syntax}
  \sx{nexpr} &::=& \sx{coeff} \sor \tk{random} \\
  &\sor& id \\
  &\sor& \sx{unop} \sx{nexpr} \sor \sx{nexpr} \sx{binop} \sx{nexpr} \\
  &\sor& \tk{(} \sx{nexpr} \tk{)} \\
  \sx{binop} &::=& (\tk{+}\texttt{|}\tk{-}\texttt{|}\tk{*}\texttt{|}\tk{/}\texttt{|}\tk{\%})[\tk{\_}(\tk{i}\texttt{|}\tk{f}\texttt{|}\tk{d}\texttt{|}\tk{l}\texttt{|}\tk{q})[\tk{,}(\tk{n}\texttt{|}\tk{0}\texttt{|}\tk{+oo}\texttt{|}\tk{-oo}\texttt{|}\tk{?})]] \\
  \sx{unop} &::=& \tk{-} \\
  &\sor& (\pkw{cast}\texttt{|}\pkw{sqrt})[\tk{\_}(\tk{i}\texttt{|}\tk{f}\texttt{|}\tk{d}\texttt{|}\tk{l}\texttt{|}\tk{q})[\tk{,}(\tk{n}\texttt{|}\tk{0}\texttt{|}\tk{+oo}\texttt{|}\tk{-oo}\texttt{|}\tk{?})]] \\
  \sx{coeff} &::=& \sx{float} \sor \sx{rational} \\
  \sx{float} &::=& C/OCaml floating-point number syntax \\
  \sx{rational} &::=& \sx{integer} \sor \sx{integer}\tk{/}\sx{integer}
\end{syntax}
The syntax of numerical expressions is the same as the one used in
the APRON interactive toplevel (in \ocaml language).

The priorities between (Boolean and arithmetic) operators is
$\{$ \tk{*},\tk{/} $\}$
\textcolor{prog}{$>$} $\{$ \tk{+},\tk{-} $\}$
\textcolor{prog}{$>$} $\{$ \tk{<}, \tk{<=}, \tk{=}, \tk{>=},\tk{>} $\}$
\textcolor{prog}{$>$} $\{$ \kw{not} $\}$
\textcolor{prog}{$>$} $\{$ \kw{and} $\}$
\textcolor{prog}{$>$} $\{$ \kw{or} $\}$

By default, numerical expressions are evaluated using exact
arithmetic on real numbers (even if some involved variables are
integer), and when assigned to a variable, filtered by the
variable domain (when it is integers). For instance, the
assignement
\begin{prog}
  x = 3/4;
\end{prog}
\noindent is equivalent to the \texttt{halt} instruction if
\texttt{x} is declared as integer, because the rational 3/4 is not
an integer value. The problem here is that the instruction is not
well-typed (even if this is not checked). If one want to emulate
the semantics of the integer division operation, with rounding to
zero, one should write
\begin{prog}
  x = 3 /_i,0 4;
\end{prog}

Indeed, most arithmetic operators have optional qualifiers that
allows to modify their semantics. The first qualifier indicates a
numerical type, the second one the rounding mode.
\begin{itemize}
\item By default, operations have an exact arithmetic semantics in
  the real numbers (even if involved variables are of integer).

  The type qualifiers modify this default semantics.
  Their meaning is as follows:
  \begin{quote}
    \begin{tabular}{ll}
      i & integer semantics \\
      f & IEEE754 32 bits floating-point semantics \\
      d & IEEE754 64 bits floating-point semantics \\
      l & IEEE754 80 bits extended floating-point semantics \\
      q & IEEE754 128 bits floating-point semantics \\
    \end{tabular}
  \end{quote}

\item By default, the rounding mode is ``any'' (this applies only
  in non-real semantics), which allows to emulate all the
  following rounding modes:
  \begin{quote}
    \begin{tabular}{ll}
      n & nearest \\
      0 & towards zero \\
      +oo & towards infinity \\
      -oo & towards minus infinity \\
      ? & emulates any rounding mode
    \end{tabular}
  \end{quote}
\end{itemize}

\paragraph{Examples of numerical expressions.}\T~

We illustrate the syntax and (approximated) semantics of numerival
expressions with the following program, annotated by Interproc
using the command
\begin{example}
  interproc.opt -domain polka -colorize false examples/numerical.spl  
\end{example}

{
\begin{prog}\footnotesize
proc integer () returns () var a : int, b : int, c : int;
begin
  /* (L5 C5) top */
  a = 5; /* (L6 C8) \textcolor{red}{[|a-5=0|]} */
  b = 2; /* (L7 C8) \textcolor{red}{[|b-2=0; a-5=0|]} */
  if brandom then
     /* (L8 C17) \textcolor{red}{[|b-2=0; a-5=0|]} */
     c = a / b; /* (L9 C14) \textcolor{red}{bottom} */
  else
    /* (L10 C6) \textcolor{red}{[|b-2=0; a-5=0|]} */
    c = a /_i,? b; /* (L11 C16) \textcolor{red}{[|b-2=0; a-5=0; -c+3>=0; c-2>=0|]} */
    c = a /_i,0 b; /* (L12 C18) \textcolor{red}{[|c-2=0; b-2=0; a-5=0|]} */
    c = a /_i,-oo b; /* (L13 C20) \textcolor{red}{[|c-2=0; b-2=0; a-5=0|]} */
    c = a /_i,+oo b; /* (L14 C20) \textcolor{red}{[|c-3=0; b-2=0; a-5=0|]} */
    c = a /_i,n b; /* (L15 C18) \textcolor{red}{[|b-2=0; a-5=0; -c+3>=0; c-2>=0|]} */
    c = a \%_i,? b; /* (L16 C16) \textcolor{red}{[|c-1=0; b-2=0; a-5=0|]} */
  endif; /* (L17 C8) \textcolor{red}{[|c-1=0; b-2=0; a-5=0|]} */
end

proc exact () returns (z : real) var x : real, y : real;
begin
  /* (L22 C5) \textcolor{red}{top} */
  x = 5; /* (L23 C8) \textcolor{red}{[|x-5=0|]} */
  y = 2; /* (L24 C8) \textcolor{red}{[|y-2=0; x-5=0|]} */
  z = x / y; /* (L25 C10) \textcolor{red}{[|2z-5=0; y-2=0; x-5=0|]} */
  y = 0.1; /* (L26 C10)
              \textcolor{red}{[|2z-5=0; 36028797018963968y-3602879701896397=0; x-5=0|]} */
  z = x + y; /* (L27 C10)
                \textcolor{red}{[|36028797018963968z-183746864796716237=0;
                  36028797018963968y-3602879701896397=0; x-5=0|]} */
  z = z - y; /* (L28 C10)
                \textcolor{red}{[|z-5=0; 36028797018963968y-3602879701896397=0; x-5=0|]} */
end

proc floating () returns (z : real) var x : real, y : real;
begin
  /* (L33 C5) \textcolor{red}{top} */
  x = 5; /* (L34 C8) \textcolor{red}{[|x-5=0|]} */
  y = 2; /* (L35 C8) \textcolor{red}{[|y-2=0; x-5=0|]} */
  z = x /_f,? y; /* (L36 C13)
                    \textcolor{red}{[|y-2=0; x-5=0;
                      -713623846352979940529142984724747568191373312z
                      +1784059828558929176909397126420998565333565441>=0;
                      713623846352979940529142984724747568191373312z
                      -1784059403205970525736317797202739275623301119>=0|]} */
  y = 0.1; /* (L37 C10)
              \textcolor{red}{[|36028797018963968y-3602879701896397=0; x-5=0;
                -713623846352979940529142984724747568191373312z
                +1784059828558929176909397126420998565333565441>=0;
                713623846352979940529142984724747568191373312z
                -1784059403205970525736317797202739275623301119>=0|]} */
  z = x +_f,? y; /* (L38 C14)
                    \textcolor{red}{[|36028797018963968y-3602879701896397=0; x-5=0;
                      -713623846352979940529142984724747568191373312z
                      +3639482050260215524856578735848702239922192385>=0;
                      713623846352979940529142984724747568191373312z
                      -3639481182540179876463495959770156714984210431>=0|]} */
  z = z -_f,? y; /* (L39 C14)
                    \textcolor{red}{[|36028797018963968y-3602879701896397=0; x-5=0;
                      -5986310706507378352962293074805895248510699696029696z
                      +29931560882862943060522688904967984086376852368654337>=0;
                      5986310706507378352962293074805895248510699696029696z
                      -29931546182211708189135890236173744477275669529624577>=0|]} */
end

var z : real
begin
  /* (L43 C5) \textcolor{red}{top} */
  () = integer(); /* (L44 C17) \textcolor{red}{top} */
  z = exact(); /* (L45 C14) \textcolor{red}{[z-5=0|]} */
  z = floating(); /* (L46 C17)
                     \textcolor{red}{[|-5986310706507378352962293074805895248510699696029696z
                       +29931560882862943060522688904967984086376852368654337>=0;
                       5986310706507378352962293074805895248510699696029696z
                       -29931546182211708189135890236173744477275669529624577>=0|]} */
end
\end{prog}
}

% ======================================================================
\subsection{Formal semantics}
% ======================================================================

We give here the standard operational semantics of the ``Simple''
language.

% ----------------------------------------------------------------------
\subsubsection{Semantics domains.}
% ----------------------------------------------------------------------

They are summarized by the following table:
\htmlonly{\begin{quote}}
\begin{image}{semdom}
  \textcolor{black}{
    \begin{tabular}{ll}
      $k\in K$ & program's control points \\
      $\epsilon \in \mathrm{Env} ~=~ \mathrm{Var}\rightarrow \mathbb{R}$ &
      environments \\
      $(k,\epsilon)\in K\times\mathrm{Env}$ &
      activation records \\
      $s\in S=(K\times \mathrm{Env})^+$ &
      program states, which are non-empty stack of activation records
    \end{tabular}
}
  \end{image}
\htmlonly{\end{quote}}

An environment is defined by a function assigning a value to a set
of variables $\mathrm{Var}$ visible in the current scope.

A program state is a non-empty stack of activation records. A
stack is viewed as a word where the last letter corresponds to the
top of the stack.

% ----------------------------------------------------------------------
\subsubsection{Semantics of expressions.}
% ----------------------------------------------------------------------

The semantics of a numerical expression \texttt{nexpr}
is a function
\htmlonly{\begin{quote}}\begin{image}{nexpr}
  $$
  \sem{\texttt{nexpr}}^n : \mathrm{Env} \rightarrow \wp(\mathbb{R})
  $$
 \end{image}\htmlonly{\end{quote}}
We refer to \cite{mine04} for the precise definition of this semantic
function.

The semantics of a Boolean expression \texttt{bexpr}
is a function %
\htmlonly{\begin{quote}}\begin{image}{bexpr0}
  $$
  \sem{\texttt{bexpr}}^b : \mathrm{Env} \rightarrow \wp(\mathbb{B})
  $$
 \end{image}\htmlonly{\end{quote}}
which is defined inductively as follows:
\htmlonly{\begin{quote}}\begin{image}{bexpr1}
  $$
  \begin{array}{c@{~=~}l}
    \sem{\texttt{true}}^b & \lambda \epsilon~.~\{ true \} \\
    \sem{\texttt{false}}^b & \lambda \epsilon~.~\{ false \} \\
    \sem{\texttt{brandom}}^b & \lambda \epsilon~.~\{ true,false \} \\
    \sem{\texttt{not~} e}^b & \lambda \epsilon~.~\{ \neg b ~|~ b \in \sem{e}^b(\epsilon) \} \\
    \sem{e_1 \texttt{~and~} e_2}^b & \lambda \epsilon~.~\{ b_1\wedge b_2 ~|~ b_1 \in \sem{e_1}^b(\epsilon) \wedge b_2 \in \sem{e_2}^b(\epsilon)\} \\
    \sem{e_1 \texttt{~or~} e_2}^b & \lambda \epsilon~.~\{ b_1\vee b_2 ~|~ b_1 \in \sem{e_1}^b(\epsilon) \wedge b_2 \in \sem{e_2}^b(\epsilon)\} \\
    \sem{e_1 \texttt{~op~} e_2}^b & \sem{(e_1 - e_2) \texttt{~op~} 0}^b \\
    \sem{e=0}^b & \lambda \epsilon ~.~ (0 \in \sem{e}^n(\epsilon)) \\
    \sem{e\geq 0}^b & \lambda \epsilon ~.~ (\sem{e}^n(\epsilon)\cap [0,\infty[ \neq \emptyset) \\
    \sem{e\leq 0}^b & \lambda \epsilon ~.~ (\sem{e}^n(\epsilon)\cap ]-\infty,0] \neq \emptyset) \\
    \sem{e> 0}^b & \lambda \epsilon ~.~ (\sem{e}^n(\epsilon)\cap ]0,\infty] \neq \emptyset) \\
    \sem{e< 0}^b & \lambda \epsilon ~.~ (\sem{e}^n(\epsilon)\cap ]-\infty,0[ \neq \emptyset)
  \end{array}
  $$
 \end{image}\htmlonly{\end{quote}}
The semantics of a Boolean expression can be extended as a predicate on environments as follows:
\htmlonly{\begin{quote}}\begin{image}{bexpr2}
  $$
    \sem{\texttt{bexpr}}^b ~:
    \begin{array}[t]{ccl}
      \mathrm{Env} &\rightarrow& \mathbb{B} \\
      \epsilon &\mapsto& (true ~\in~ \sem{\texttt{bexpr}}^b)
    \end{array}
  $$
 \end{image}\htmlonly{\end{quote}}

% ----------------------------------------------------------------------
\subsubsection{Semantics of programs}
% ----------------------------------------------------------------------

We define the semantics of a program as a discrete dynamical
system $(S,S_0,\rightarrow)$ where $S$ is the state-space of the
program (see above), $S_0$ is the set of initial states, and
$\rightarrow$ is the transition relation between states.

If $k_0$ denotes the starting point of the main procedure, then $S_0$ is defined as
\htmlonly{\begin{quote}}\begin{image}{S0}
  $$
    S_0=\{k_0\} \times \mathrm{Env}
  $$
 \end{image}\htmlonly{\end{quote}}
This reflects the hypothesis that variables are uninitialized
(read: they may contain arbitrary values) at start.

The transition relation is defined as follows:

\W\htmlonly{\begin{quote}}
\begin{image}{trans}
  $$
  \begin{array}{c}
    (k) \texttt{~skip~} (k') \\ \hline
    \omega\cdot (k,\epsilon) \rightarrow \omega\cdot (k',\epsilon)
  \end{array}
  \qquad
  \begin{array}{c}
    (k) \texttt{~assume~} bexpr ~ (k') \quad true\in\sem{bexpr}^b(\epsilon) \\ \hline
    \omega\cdot (k,\epsilon) \rightarrow \omega\cdot (k',\epsilon)
  \end{array}
  $$
  $$
  \begin{array}{c}
    (k) \texttt{~x=random~} (k') \quad \forall y\in\mathrm{Var}\setminus\{x\}:\epsilon'(y)=\epsilon(y) \\ \hline
    \omega\cdot (k,\epsilon) \rightarrow \omega\cdot (k',\epsilon')
  \end{array}
  $$
  $$
  \begin{array}{c}
    (k) \texttt{~x=nexpr~} (k') \quad v\in\sem{nexpr}^n(\epsilon)  \\ \hline
    \omega\cdot (k,\epsilon) \rightarrow \omega\cdot (k',\epsilon[x\mapsto v])
  \end{array}
  $$
  $$
  \begin{array}{c}
    (k) \texttt{~if bexpr then~} (k_1)\ldots(k'_1) \texttt{~else~} (k_2)\ldots (k'_2) \texttt{~;~} (k')\\ \hline
    (k,\epsilon) \rightarrow (k_1,\epsilon) \texttt{~if~} true\in\sem{bexpr}^b(\epsilon) \\
    (k,\epsilon) \rightarrow (k_2,\epsilon) \texttt{~if~} false\in\sem{bexpr}^b(\epsilon) \\
    (k'_1,\epsilon) \rightarrow (k',\epsilon) \\
    (k'_2,\epsilon) \rightarrow (k',\epsilon)
  \end{array}
  $$
  $$
  \begin{array}{c}
    (k) \texttt{~while bexpr do~} (k_1)\ldots(k'_1) \texttt{~done~} (k_2)\\ \hline
    (k,\epsilon) \rightarrow (k_1,\epsilon) \texttt{~if~} true\in\sem{bexpr}^b(\epsilon) \\
    (k,\epsilon) \rightarrow (k_2,\epsilon) \texttt{~if~} false\in\sem{bexpr}^b(\epsilon) \\
    (k'_1,\epsilon) \rightarrow (k,\epsilon)
  \end{array}
  $$
  $$
  \begin{array}{c}
    (k) \texttt{~(y0,...,yN)=f(x0,...xM)~} (k') \\
    \epsilon_\mathrm{start}(\vec{pin}) = \epsilon(\vec{x}) \\
    \epsilon' = \epsilon[\vec{y} \mapsto \epsilon_\mathrm{exit}(\vec{pout}) \\ \hline
    \omega\cdot (k,\epsilon) \rightarrow \omega\cdot (k',\epsilon) \cdot (k^f_\mathrm{start},\epsilon_\mathrm{start}) \\
     \omega\cdot (k',\epsilon) \cdot (k^f_\mathrm{exit},\epsilon_\mathrm{exit})
     \rightarrow
     \omega\cdot (k',\epsilon')
  \end{array}
  $$
\end{image}
\W\end{quote} %

% **********************************************************************
\section{What Interproc does for you}
% **********************************************************************

Interproc takes as input a ``Simple'' program, annotates it with
control points, perform forward and/or backward analysis, and
print the annotated programs with invariants associated with each
control point, like this:
{
\begin{prog}\footnotesize
Annotated program after forward analysis:
var x : int, y : int, z : int
begin
  /* (L5 C5) \textcolor{red}{top} */
  assume z >= 10 and z <= 20; /* (L6 C25) \textcolor{red}{[|-z+20>=0; z-10>=0|]} */
  x = 0; /* (L7 C8) \textcolor{red}{[|x=0; -z+20>=0; z-10>=0|]} */
  y = 0; /* (L7 C13) \textcolor{red}{[|-3x+y=0; -x+z+1>=0; -z+20>=0; z-10>=0; x>=0|]} */
  while x <= z do
    /* (L8 C17) \textcolor{red}{[|-3x+y=0; -x+z>=0; -z+20>=0; z-10>=0; x>=0|]} */
    x = x + 1; /* (L9 C12)
                  \textcolor{red}{[|-3x+y+3=0; -x+z+1>=0; -z+20>=0; z-10>=0; x-1>=0|]} */
    y = y + 3; /* (L10 C12) \textcolor{red}{[|-3x+y=0; -x+z+1>=0; -z+20>=0; z-10>=0; x-1>=0|]} */
  done; /* (L11 C7) \textcolor{red}{[|-3x+y=0; -x+z+1=0; -x+21>=0; x-11>=0|]} */
  if y >= 42 then
    /* (L12 C15) \textcolor{red}{[|-3x+y=0; -x+z+1=0; -x+21>=0; x-14>=0|]} */
    fail; /* (L13 C9) \textcolor{red}{bottom} */
  endif; /* (L14 C8) \textcolor{red}{[|-3x+y=0; -x+z+1=0; -3x+41>=0; x-11>=0|]} */
end
\end{prog}
}

% ----------------------------------------------------------------------
\subsubsection*{Invariants}
% ----------------------------------------------------------------------

An \emph{invariant} $I$ is a set of environments:
\htmlonly{\begin{quote}}\begin{image}{inv1}
  $$
I\in\wp(\mathrm{Env})
  $$
\end{image}\htmlonly{\end{quote}}
The result of the analyses performed by Interproc is the
projection $p(X)$ of sets of stacks on their top elements. The result of this projection is a function that maps control point to invariants:
\htmlonly{\begin{quote}}\begin{image}{inv2}
  $$
p~ :
\begin{array}[t]{ccl}
\wp((K\times\mathrm{Env})^+) &\longrightarrow& (K\rightarrow \wp(\mathrm{Env}) \\
X &\longmapsto& \lambda k~.~ \{ \epsilon ~|~ \omega\cdot (k,\epsilon)\in X \}
\end{array}
  $$
 \end{image}\htmlonly{\end{quote}}

Intuitively, the invariant on numerical variables associated with
each program point, as returned by Interproc, indicates the
possible values for these variables at each program point, for any
stack tail.

% ----------------------------------------------------------------------
\subsubsection*{Single forward/reachability analysis}
% ----------------------------------------------------------------------

A forward analysis compute the set of reachable states of the program, which is defined as:
\htmlonly{\begin{quote}}\begin{image}{reach0}
  $$
Reach = \{ s\in S ~|~ s_0\in S_0 \wedge s_0\rightarrow^* s \}
  $$
\end{image}\htmlonly{\end{quote}}
Interproc returns an overapproximation of $p(Reach)$.

% ----------------------------------------------------------------------
\subsubsection*{Single backward/coreachability analysis}
% ----------------------------------------------------------------------

A backward analysis computes the set of states from which one
can reach a \texttt{fail} instruction.

The set of states coreachable from a \texttt{fail} instruction is
defined as: 
\htmlonly{\begin{quote}}\begin{image}{coreach0}
$$
Coreach = \{ s\in S ~|~ s\rightarrow^* s_f \wedge s_f\in F \}
$$
with the set of final states 
$$
F = \{ \omega\cdot (k,\epsilon) ~|~ k\in K_\mathrm{fail} \wedge \epsilon\in\mathrm{Env} \}
$$
where $K_\mathrm{fail}$ is the set of control points just
preceding \texttt{fail} instructions.
\end{image}\htmlonly{\end{quote}}
Interproc returns an overapproximation of $p(Coreach)$.

% ----------------------------------------------------------------------
\subsubsection*{Combined forward/backward analyses}
% ----------------------------------------------------------------------

Interproc allows to combine forward and backward analysis any
number of time.

If $X$ denotes the invariant computed by Interproc at the
previous step, at the current step:
\begin{itemize}
\item a forward analysis will consider the set of
  initial states
  \htmlonly{\begin{quote}}\begin{image}{reach1}
    $$
    S_0 \cap p^{-1}(X)
    $$
   \end{image}\htmlonly{\end{quote}}
\item a backward analysis will consider the set of
  final states
  \htmlonly{\begin{quote}}\begin{image}{coreach1}
    $$
    F \cap p^{-1}(X)
    $$
   \end{image}\htmlonly{\end{quote}}
\end{itemize}

For instance, after executing the command
\begin{example}
  interproc -analysis fb essai.spl
\end{example}
on the previous program, we get
{
\begin{prog}\footnotesize
Annotated program after forward analysis
var x : int, y : int, z : int
begin
  /* (L5 C5) \textcolor{red}{top} */
  assume z >= 10 and z <= 20; /* (L6 C25) \textcolor{red}{[|-z+20>=0; z-10>=0|]} */
  x = 0; /* (L7 C8) \textcolor{red}{[|x=0; -z+20>=0; z-10>=0|]} */
  y = 0; /* (L7 C13) \textcolor{red}{[|-3x+y=0; -x+z+1>=0; -z+20>=0; z-10>=0; x>=0|]} */
  while x <= z do
    /* (L8 C17) \textcolor{red}{[|-3x+y=0; -x+z>=0; -z+20>=0; z-10>=0; x>=0|]} */
    x = x + 1; /* (L9 C12)
                  \textcolor{red}{[|-3x+y+3=0; -x+z+1>=0; -z+20>=0; z-10>=0; x-1>=0|]} */
    y = y + 3; /* (L10 C12) \textcolor{red}{[|-3x+y=0; -x+z+1>=0; -z+20>=0; z-10>=0; x-1>=0|]} */
  done; /* (L11 C7) \textcolor{red}{[|-3x+y=0; -x+z+1=0; -x+21>=0; x-11>=0|]} */
  if y >= 42 then
    /* (L12 C15) \textcolor{red}{[|-3x+y=0; -x+z+1=0; -x+21>=0; x-14>=0|]} */
    fail; /* (L13 C9) \textcolor{red}{bottom} */
  endif; /* (L14 C8) \textcolor{red}{[|-3x+y=0; -x+z+1=0; -3x+41>=0; x-11>=0|]} */
end

Annotated program after backward analysis
var x : int, y : int, z : int
begin
  /* (L5 C5) \textcolor{red}{[|-z+20>=0; z-13>=0|]} */
  assume z >= 10 and z <= 20; /* (L6 C25) \textcolor{red}{[|-z+20>=0; z-13>=0|]} */
  x = 0; /* (L7 C8) \textcolor{red}{[|x=0; -z+20>=0; z-13>=0|]} */
  y = 0; /* (L7 C13) \textcolor{red}{[|-3x+y=0; -x+z+1>=0; -z+20>=0; z-13>=0; x>=0|]} */
  while x <= z do
    /* (L8 C17) \textcolor{red}{[|-3x+y=0; -x+z>=0; -z+20>=0; z-13>=0; x>=0|]} */
    x = x + 1; /* (L9 C12)
                  \textcolor{red}{[|-3x+y+3=0; -x+z+1>=0; -z+20>=0; z-13>=0; x-1>=0|]} */
    y = y + 3; /* (L10 C12) \textcolor{red}{[|-3x+y=0; -x+z+1>=0; -z+20>=0; z-13>=0; x-1>=0|]} */
  done; /* (L11 C7) \textcolor{red}{[|-3x+y=0; -x+z+1=0; -x+21>=0; x-14>=0|]} */
  if y >= 42 then
    /* (L12 C15) \textcolor{red}{[|-3x+y=0; -x+z+1=0; -x+21>=0; x-14>=0|]} */
    fail; /* (L13 C9) \textcolor{red}{bottom} */
  endif; /* (L14 C8) \textcolor{red}{bottom} */
end
\end{prog}
}
\bibliographystyle{plain}
\bibliography{manual}

\end{document}
